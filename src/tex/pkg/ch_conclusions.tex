\chapter{Conclusions}

We described in depth the implementation of
the merge algorithm as formulated by A. Paoluzzi et al.
\cite{Paoluzzi}. We introduced the thesis using a very brief
theoretical overview of the algorithm and its applications,
and then we explored the implementation starting from
the $d=3$ case down to the $d=1$ one. At the end we presented
two very representative examples to show the capabilities of this
algorithm.

\section{Future developments}

\subsection{Parallelization}
In the introduction, we said the algorithm follows 
the \textit{divide et impera} philosophy; this happens
to be a very good thing to do while doing parallel programming.
The best way to parallelize the system is to launch
a separate job for every face of the 2-skeleton during the
fragmentation of the complex, which is the heavier part of
the whole algorithm.
For this reason, the implementation has been developed to
be ``parallel ready'', in this way, it will be possible
to easily make this implementation parallel for real using
the Julia parallelization capabilities. This work will
be done in the next few weeks.

\subsection{Boolean operations}
As you may have noticed, the arrangement algorithm
is not enough to perform Boolean operations by itself.
But you may also have noticed we developed this implementation
being constantly conscious about the finalities of the algorithm
and so we laid the foundations to a future easy implementation
of real Boolean operations.

\subsection{Handling of $3$-cells with non-intesecting shells}
In this implementation we handled only $2$-cells with non-intersecting
shells [ref. \ref{sec:planar_arrangement_overview}] but also $3$-cells
with non-intersecting shells (which trivially are 3-cells with holes) 
must be handled. This lack will be
fixed as soon as possible following the directions of the ALGORITHM 2
as described by A. Paoluzzi et al. \cite{Paoluzzi}.

\subsection{LAR}
This thesis is only the beginning of the Julia implementation
of LAR. There are many modules that has been written 
during the development of the Python version of LAR
[ref. \ref{sec:history}] that are ready to be ported
on the foundations that the module presented in this thesis
laid. The porting of these other modules is planned for the
next months.