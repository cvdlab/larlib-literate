\chapter{Spatial Arrangement}

%%%%%%%%%%%%%%%%%%
\section{Overview}
Here we present the spatial arrangement algorithm.
It has been explained in the introduction [ref. \ref{sec:spatial_arrangement_overview}].

@O lib/jl/spatial_arrangement.jl
@{@< spatial\_arrangement support functions @>

function spatial_arrangement(V::Verts, EV::Cells, FE::Cells)
    vs_num = size(V, 1)
    es_num = size(EV, 1)
    fs_num = size(FE, 1)

    sp_idx = spatial_index(V, EV, FE)

    rV = Verts(0,3)
    rEV = spzeros(Int8,0,0)
    rFE = spzeros(Int8,0,0)

    for sigma in 1:fs_num
        print(sigma, "/", fs_num, "\r")

        sigmavs = (abs.(FE[sigma:sigma,:])*abs.(EV))[1,:].nzind 
        sV = V[sigmavs, :]
        sEV = EV[FE[sigma, :].nzind, sigmavs]

        @< Sigma flattening @>

        nV, nEV, nFE = planar_arrangement(sV, sEV, sparsevec(ones(Int8, length(sigmavs))))

        if nV == nothing
            continue
        end
        
        nvsize = size(nV, 1)
        nV = [nV zeros(nvsize) ones(nvsize)]*inv(M)[:, 1:3]

        rV, rEV, rFE = skel_merge(rV, rEV, rFE, nV, nEV, nFE)
    end

    rV, rEV, rFE = merge_vertices(rV, rEV, rFE)

    @< Create 3-cells @>

    return rV, rEV, rFE, rCF
end

@}

To flatten the 2-cell $\sigma$ on the $x_3=0$ plane,
we build a linear transformation matrix with the
\texttt{submanifold\_mapping} utility [ref. \ref{sec:submanifold_mapping}],
we transform the geometry and we intersect every cell in $\mathcal{I}(\sigma)$
[ref. \ref{sec:spatial_arrangement_overview}]
with the $x_3=0$ plane using \texttt{face\_int} [ref. \ref{sec:face_int}].

@D Sigma flattening
@{M = submanifold_mapping(sV)
tV = ([V ones(vs_num)]*M)[:, 1:3]

sV = tV[sigmavs, :]

for i in sp_idx[sigma]
    tmpV, tmpEV = face_int(tV, EV, FE[i, :])
    
    sV, sEV = skel_merge(sV, sEV, tmpV, tmpEV)
end

sV = sV[:, 1:2]
@}



%%%%%%%%%%%%%%%%%%%%%%%%%%%%%%%%%%%
\section{Coincident vertices merge}
\label{sec:3D_merge_vertices}

The merge of coincident is done in the \texttt{merge\_vertices}
function.

@D spatial\_arrangement support functions
@{function merge_vertices(V::Verts, EV::Cells, FE::Cells, err=1e-4)
    vertsnum = size(V, 1)
    edgenum = size(EV, 1)
    facenum = size(FE, 1)
    newverts = zeros(Int, vertsnum)
    kdtree = KDTree(V')

    @< Find coincident vertices @>
    @< Merge edges @>
    @< Merge faces @>

    return nV, nEV, nFE
end
@}

First of all we need to find vertices which are near enough
to be considered coincident. We perform this operation
relying on the \texttt{NearestNeighbors.jl} package\cite{NearestNeighbors}
which provides a rather good implementation of the \texttt{KDTree} data structure.

So, we identify the vertices to delete and we store a map
from original vertices to new vertices. In the meanwhile
we built a list of vertices to delete and we delete them 
as soon as possible.

@D LAR imports
@{using NearestNeighbors
@}

@D Find coincident vertices
@{todelete = []

i = 1
for vi in 1:vertsnum
    if !(vi in todelete)
        nearvs = inrange(kdtree, V[vi, :], err)

        newverts[nearvs] = i

        nearvs = setdiff(nearvs, vi)
        todelete = union(todelete, nearvs)

        i = i + 1
    end
end

nV = V[setdiff(collect(1:vertsnum), todelete), :]
@}

To delete the edges we write them as couples of vertex
indices. We keep them in two versions: in \texttt{edges}
we put the edges described with the indexes of the new vertices
and in \texttt{oedges} we put the edges relative to the original vertex
indices (we will use them when merging faces). Once we "translated"
the edges, we delete the duplicates (using a set union) and the
degenerated edges. Lastly we build a new EV matrix 
(called \texttt{nEV}). While we
build the matrix, we also build a dictionary which maps edges expressed
as couples of vertex indices into edge indices relative to \texttt{nEV};
this data will be used in the $d=2$ version of this function 
[ref. \ref{sec:2D_merge_vertices}].

@D Merge edges
@{edges = Array{Tuple{Int, Int}, 1}(edgenum)
oedges = Array{Tuple{Int, Int}, 1}(edgenum)

for ei in 1:edgenum
    v1, v2 = EV[ei, :].nzind
    
    edges[ei] = Tuple{Int, Int}(sort([newverts[v1], newverts[v2]]))
    oedges[ei] = Tuple{Int, Int}(sort([v1, v2]))

end
nedges = union(edges)
nedges = filter(t->t[1]!=t[2], nedges)

nedgenum = length(nedges)
nEV = spzeros(Int8, nedgenum, size(nV, 1))

etuple2idx = Dict{Tuple{Int, Int}, Int}()

for ei in 1:nedgenum
    nEV[ei, collect(nedges[ei])] = 1
    etuple2idx[nedges[ei]] = ei
end
@}

To merge the faces, we convert them into a lists of
edges (represented as a couple of vertices). We then remove duplicated faces
by checking which faces use the same vertices. At the end, we use the
maps built during vertices and edges merge to rebuild the \texttt{FE}
matrix correctly using the new vertex indices.

@D Merge faces
@{faces = [[
    map(x->newverts[x], FE[fi, ei] > 0 ? oedges[ei] : reverse(oedges[ei]))
    for ei in FE[fi, :].nzind
] for fi in 1:facenum]


visited = []
function filter_fn(face)

    verts = []
    map(e->verts = union(verts, collect(e)), face)
    verts = Set(verts)

    if !(verts in visited)
        push!(visited, verts)
        return true
    end
    return false
end

nfaces = filter(filter_fn, faces)

nfacenum = length(nfaces)
nFE = spzeros(Int8, nfacenum, size(nEV, 1))

for fi in 1:nfacenum
    for edge in nfaces[fi]
        ei = etuple2idx[Tuple{Int, Int}(sort(collect(edge)))]
        nFE[fi, ei] = sign(edge[2] - edge[1])
    end
end
@}




%%%%%%%%%%%%%%%%%%%%%%%%%%%%%%%%%%%
\section{$3$-cells creation}

@D Create 3-cells
@{@< Compute connected components @>
@< Compute containment tree @>
@< Cell union @>

rCF = minimal_3cycles(rV, rEV, rFE)
@}