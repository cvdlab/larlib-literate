\chapter{Module overview}



%===============================================================================
\section{Data transforms and boundary operators}\label{sec:boundary}
%===============================================================================

In this section we provide some Julia functions to compute (boundary) operator representations starting from other data representations, i.e., generally speaking, starting from one/two incidence relations.

\paragraph{Sparse characteristic matrices $M_p$}

The generic function \texttt{characteristicMatrix} produces a compressed sparse column representation of a single incidence relation, i.e.~the associated characteristic matrix $M_p$ In this code we use $p=2$ and \texttt{FV} as parameter name, but it may refer to any other incidence relation as actual value. The function employs a COO (coordinate) method to build the sparse matrix, i.e.~starts from a triple of arrays for \texttt{I,J,V} values.


%-------------------------------------------------------------------------------
@D Computation of sparse characteristic matrices $M_p$
@{
# Characteristic matrix $M_2$, i.e. M(FV)
function characteristicMatrix(FV)
	I,J,V = Int64[],Int64[],Int8[] 
	for f=1:length(FV)
		for k in FV[f]
			push!(I,f)
			push!(J,k)
			push!(V,1)
		end
	end
	M_2 = sparse(I,J,V)
	return M_2
end
@}
%-------------------------------------------------------------------------------


Hence we get the sparse matrix \texttt{cscEV}, shown in Figure~\ref{fig:intro-0}, by 
{\small\begin{verbatim}
cscEV = characteristicMatrix(EV)
\end{verbatim}}

\paragraph{Boundary matrix $[\partial_1] = [\delta_0]^\top$}

It is worth noting that the matrix $M_1=\texttt{cscEV}$ previously computed is very similar to the operator matrix $[\delta_0]$, but with elements in $\{-1,0,1\}$ instead than in $\{0,1\}$. The computation is than easily done, by using the fact that the 1-cell is written as oriented 0-chains $\eta=\nu_k - \nu_h$, with $k>h$, and hence in coordinates we have $\texttt{spboundary1}[k,e] = 1$ and $\texttt{spboundary1}[k,e] = -1$.

%-------------------------------------------------------------------------------
@D Computation of (signed) sparse boundary $C_1 \to C_0$
@{
# Computation of sparse boundary $C_1 \to C_0$
function boundary1(EV)
	spboundary1 = characteristicMatrix(EV)'
	for e = 1:length(EV)
		spboundary1[EV[e][1],e] = -1
	end
	return spboundary1
end
@}
%-------------------------------------------------------------------------------

In this case we get the compressed sparse column matrix representation $\partial_1$, shown dense in Figure~\ref{fig:intro-2}, by computing
\[
\partial_1 = \small\texttt{boundary1(EV)}
\]
\begin{figure}[htbp] %  figure placement: here, top, bottom, or page
{\footnotesize\begin{verbatim}
julia> julia> full(boundary1(EV))
12�20 Array{Int8,2}:
 -1   0   0   0   0   0  -1   0   0   0   0   0  -1   0   0   0   0   0   0   0
  1   0   0   0   0   0   0  -1   0   0   0   0   0  -1   0   0   0   0   0   0
  0  -1   0   0   0   0   1   0   0   0   0   0   0   0  -1   0   0   0   0   0
  0   1   0   0   0   0   0   1   0   0   0   0   0   0   0  -1   0   0   0   0
  0   0  -1   0   0   0   0   0  -1   0   0   0   1   0   0   0  -1   0   0   0
  0   0   1   0   0   0   0   0   0  -1   0   0   0   1   0   0   0  -1   0   0
  0   0   0  -1   0   0   0   0   1   0   0   0   0   0   1   0   0   0  -1   0
  0   0   0   1   0   0   0   0   0   1   0   0   0   0   0   1   0   0   0  -1
  0   0   0   0  -1   0   0   0   0   0  -1   0   0   0   0   0   1   0   0   0
  0   0   0   0   1   0   0   0   0   0   0  -1   0   0   0   0   0   1   0   0
  0   0   0   0   0  -1   0   0   0   0   1   0   0   0   0   0   0   0   1   0
  0   0   0   0   0   1   0   0   0   0   0   1   0   0   0   0   0   0   0   1
\end{verbatim}}
   \caption{Dense characteristic matrix \texttt{cscFV} for the incidence \texttt{FV} of the small complex in Figure~\ref{fig:intro-1}}
   \label{fig:intro-2}
\end{figure}


\paragraph{Boundary matrix $[\partial_2] = [\delta_1]^\top$}

The matrix of the unsigned operator $\partial_2$ is computed here, according to the method introduced in~\cite{}. In particular, the two sparse incidence matrices \texttt{cscFV} and \texttt{cscEV} are first computed, and the product matrix in \texttt{temp} codifies, for any pair $(i,j)$ of indices the number of vertices shared between the  $i$-th 2-cell and the $j$-th 1-cell; when this number is equal to 2, the $j$-th 1-cell belongs to the boundary of the $i$-th 2-cell, so providing a triple $(i,j,1)$ to be inserted in the COO representation of \texttt{sp_uboundary2}, and finally in its CSC representation.

%-------------------------------------------------------------------------------
@D Computation of (unsigned) sparse boundary $C_2 \to C_1$
@{
# Computation of sparse uboundary2
function uboundary2(FV,EV)
	cscFV = characteristicMatrix(FV)
	cscEV = characteristicMatrix(EV)
	temp = cscFV * cscEV'
	I,J,V = Int64[],Int64[],Int8[]
	for j=1:size(temp,2)
		for i=1:size(temp,1)
			if temp[i,j] == 2
    			push!(I,i)
    			push!(J,j)
    			push!(V,1)
			end
		end
	end
	sp_uboundary2 = sparse(I,J,V)
	return sp_uboundary2
end
@}
%-------------------------------------------------------------------------------




@O lib/jl/LARLIB.jl
@{module LARLIB

    @< LAR imports @>
    @< LAR types @>
	@< Computation of sparse characteristic matrices $M_p$ @>
	@< Computation of (signed) sparse boundary $C_1 \to C_0$ @>
	@< Computation of (unsigned) sparse boundary $C_2 \to C_1$ @>

    include("./utilities.jl")
    include("./minimal_cycles.jl")
    include("./dimension_travel.jl")
    include("./planar_arrangement.jl")
    include("./spatial_arrangement.jl")
    include("./largrid.jl")
    
end
@}

%%%%%%%%%%%%%%%
\section{Standard types}

We define at the top of our module the standard types
that will be used throughout LAR. As already explained
in the introduction [ref. \ref{sec:LAR}], LAR needs
only one bi-dimensional array to store geometry and 
one or more sparse matrices for topology.
Julia has already implemented CSC sparse matrices in
its standard library so we are going to use them.

@D LAR types
@{const Verts = Array{Float64, 2}
const Cells = SparseMatrixCSC{Int8, Int}
const Cell = SparseVector{Int8, Int}
const LarCells = Array{Array{Int, 1}, 1}
@}

We used the general name \texttt{Cells}, but
we are going to use this type also for boundaries.

\subsection{Floating point error}
\label{sec:floating-point_error}

We stored geometry using 64-bit IEEE floats.
As it is known, floating point arithmetic is not
precise and introduces numerical errors.
Usually this is not an issue\footnote{The \textit{machine epsilon},
which is the upper bound on the relative error in floating-point 
arithmetic, for double precision IEEE floating-point numbers is 
$2^53 \approx 1.11 \times 10^{-16}$.}, but when precision is
a goal, floating point error must be handled very carefully.
During the development we encountered several numerical problems
and we tried various approaches (like normalizing the geometry
inside the $[0, 1]$ interval for each dimension in order to maximize
the significand of the floating-point numbers) but most of them turned 
out to be unstable. So we choose the less orthodox path we could
possibly take: we set a fixed error and we performed every floating point
comparison using this error. Examples of this ``tweak'' are to be found in
\ref{sec:intersect_edges}, \ref{sec:face_int}, \ref{sec:3d_minimal_cycles} and 
\ref{sec:vertex_equality}.

%%%%%%%%%%%%%%%%%
\section{Notes on variables names}

Here a list of some often used variable names.

\begin{description}[align=right,labelwidth=2em]
    \item [\texttt{V}:]
        Bi-dimensional array (\texttt{Verts}) that keeps the geometry of a complex.
        Its dimensions are $n \times d$, where $n$ is the number of vertices and $d$ is the dimension
        of the euclidean space in which the complex is embedded.
    \item [\texttt{EV}:]
        1-boundary. It is a $m \times n$ sparse matrix (\texttt{Cells}) 
        where $m$ is the number of edges and $n$ is the number of vertices. The possible values
        are $0, 1$ and $-1$.
    \item [\texttt{FE}:]
        2-boundary. Same as \texttt{EV}, but faces on the rows and edges on the columns.
    \item [\texttt{CF}:]
        3-boundary. Same as \texttt{EV}, but 3-cells on the rows and faces on the columns.
\end{description}

%%%%%%%%%%%%%%%
\section{Tests and examples}

There are several unit tests throughout the implementation. They
are inside the \texttt{test} directory and can be run at once
by executing \texttt{test/runtests.jl}

@O test/jl/runtests.jl
@{using LARLIB

include("./planar_arrangement.jl")
include("./dimension_travel.jl")
include("./largrid.jl")
include("./utilities.jl")
@}

Also general examples of some main functionalities are provided.
They can be found into \texttt{examples/general\_examples.jl}

@O examples/jl/general_examples.jl
@{using LARLIB

@< planar\_arrangement general examples @>
@< spatial\_arrangement general examples @>
@}