\chapter{Module overview}

We structured our code in a Julia module called \texttt{LARLIB}.
We offer every function written in this thesis in the sub-module
\texttt{LARLIB.Arrangement}, but to the common user is offered
an interface of only three functions:
\begin{itemize}[noitemsep]
    \item \texttt{LARLIB.skel\_merge}: Provides the skeletal merge between two
        1-skeletons or 2-skeletons
    \item \texttt{LARLIB.spatial\_arrangement}: Arranges one 2-skeleton 
        in $\mathbb{E}^3$ passed as an array of vertices, and two boundary matrices.
    \item \texttt{LARLIB.planar\_arrangement}: Arranges one 1-skeleton 
        in $\mathbb{E}^2$ passed as an array of vertices and a boundary matrix.
\end{itemize}


@O lib/jl/LARLIB.jl
@{module LARLIB


    module Utils
        include("./utilities.jl")
    end

    module Arrangement
        using LARLIB.Utils

        include("./planar_arrangement.jl")
        include("./spatial_arrangement.jl")
    end

    function skel_merge(V1, EV1, V2, EV2)
        Arrangement.skel_merge(V1, EV1, V2, EV2)
    end

    function skel_merge(V1, EV1, FE1, V2, EV2, FE2)
        Arrangement.skel_merge(V1, EV1, FE1, V2, EV2, FE2)
    end

    function spatial_arrangement(V, EV, FE)
        Arrangement.spatial_arrangement(V, EV, FE)
    end

    function planar_arrangement(V, EV)
        Arrangement.planar_arrangement(V, EV)
    end
end
@}

%%%%%%%%%%%%%%%%%
\section{Notes on variables names}

Here a list of some often used variable names.

\begin{description}[align=right,labelwidth=2em]
    \item [\texttt{V}:]
        Bi-dimensional array (\texttt{Array\{Float64, 2\}}) that keeps the geometry of a complex.
        Its dimensions are $n \times d$, where $n$ is the number of vertices and $d$ is the dimension
        of the euclidean space in which the complex is embedded.
    \item [\texttt{EV}:]
        1-boundary. It is a $m \times n$ sparse matrix (\texttt{SparseMatrixCSC\{Int8, Int\}}) 
        where $m$ is the number of edges and $n$ is the number of vertices. The possible values
        are $0, 1$ and $-1$.
    \item [\texttt{FE}:]
        2-boundary. Same as \texttt{EV}, but faces on the rows and edges on the columns.
    \item [\texttt{CF}:]
        3-boundary. Same as \texttt{EV}, but 3-cells on the rows and faces on the columns.
    

\end{description}